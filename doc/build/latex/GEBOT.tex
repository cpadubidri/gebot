%% Generated by Sphinx.
\def\sphinxdocclass{report}
\documentclass[letterpaper,10pt,english]{sphinxmanual}
\ifdefined\pdfpxdimen
   \let\sphinxpxdimen\pdfpxdimen\else\newdimen\sphinxpxdimen
\fi \sphinxpxdimen=.75bp\relax
\ifdefined\pdfimageresolution
    \pdfimageresolution= \numexpr \dimexpr1in\relax/\sphinxpxdimen\relax
\fi
%% let collapsible pdf bookmarks panel have high depth per default
\PassOptionsToPackage{bookmarksdepth=5}{hyperref}

\PassOptionsToPackage{booktabs}{sphinx}
\PassOptionsToPackage{colorrows}{sphinx}

\PassOptionsToPackage{warn}{textcomp}
\usepackage[utf8]{inputenc}
\ifdefined\DeclareUnicodeCharacter
% support both utf8 and utf8x syntaxes
  \ifdefined\DeclareUnicodeCharacterAsOptional
    \def\sphinxDUC#1{\DeclareUnicodeCharacter{"#1}}
  \else
    \let\sphinxDUC\DeclareUnicodeCharacter
  \fi
  \sphinxDUC{00A0}{\nobreakspace}
  \sphinxDUC{2500}{\sphinxunichar{2500}}
  \sphinxDUC{2502}{\sphinxunichar{2502}}
  \sphinxDUC{2514}{\sphinxunichar{2514}}
  \sphinxDUC{251C}{\sphinxunichar{251C}}
  \sphinxDUC{2572}{\textbackslash}
\fi
\usepackage{cmap}
\usepackage[T1]{fontenc}
\usepackage{amsmath,amssymb,amstext}
\usepackage{babel}



\usepackage{tgtermes}
\usepackage{tgheros}
\renewcommand{\ttdefault}{txtt}



\usepackage[Bjarne]{fncychap}
\usepackage{sphinx}

\fvset{fontsize=auto}
\usepackage{geometry}


% Include hyperref last.
\usepackage{hyperref}
% Fix anchor placement for figures with captions.
\usepackage{hypcap}% it must be loaded after hyperref.
% Set up styles of URL: it should be placed after hyperref.
\urlstyle{same}

\addto\captionsenglish{\renewcommand{\contentsname}{Contents:}}

\usepackage{sphinxmessages}
\setcounter{tocdepth}{1}



\title{GEBOT Documentation}
\date{Dec 20, 2023}
\release{1.0.0}
\author{Chirag Padubidri}
\newcommand{\sphinxlogo}{\vbox{}}
\renewcommand{\releasename}{Release}
\makeindex
\begin{document}

\ifdefined\shorthandoff
  \ifnum\catcode`\=\string=\active\shorthandoff{=}\fi
  \ifnum\catcode`\"=\active\shorthandoff{"}\fi
\fi

\pagestyle{empty}
\sphinxmaketitle
\pagestyle{plain}
\sphinxtableofcontents
\pagestyle{normal}
\phantomsection\label{\detokenize{index::doc}}



\chapter{Indices and tables}
\label{\detokenize{index:indices-and-tables}}\begin{itemize}
\item {} 
\sphinxAtStartPar
\DUrole{xref,std,std-ref}{genindex}

\item {} 
\sphinxAtStartPar
\DUrole{xref,std,std-ref}{modindex}

\item {} 
\sphinxAtStartPar
\DUrole{xref,std,std-ref}{search}

\end{itemize}
\phantomsection\label{\detokenize{index:module-gebot.ImageDownloader}}\index{module@\spxentry{module}!gebot.ImageDownloader@\spxentry{gebot.ImageDownloader}}\index{gebot.ImageDownloader@\spxentry{gebot.ImageDownloader}!module@\spxentry{module}}
\sphinxAtStartPar
\sphinxstylestrong{IMAGE DOWNLOADER}

\sphinxAtStartPar
ImageDownloader is a class designed for downloading Google Earth images based on coordinates (Longitude, Latitude) retrieved from a CSV database. The images are centered around the specified coordinates.

\sphinxAtStartPar
To set up the bot, ensure that the ‘getLoc.py’ script has been executed, and the ‘config.json’ file is located in the ‘./resources’ folder. The ‘config\_path’ parameter in the constructor allows customization of the configuration file path.

\sphinxAtStartPar
In the current version, the bot is equipped with download logic, a notification handler (to send the status to the registered email on failure), and a status window (to display the bot’s status).
\begin{quote}\begin{description}
\sphinxlineitem{param config\_path}
\sphinxAtStartPar
The file path to the configuration file (default is ‘./resources/config.json’).

\sphinxlineitem{type config\_path}
\sphinxAtStartPar
str, optional

\end{description}\end{quote}
\subsubsection*{Examples}

\sphinxAtStartPar
Initialize the ImageDownloader with a custom configuration file path:

\begin{sphinxVerbatim}[commandchars=\\\{\}]
\PYG{g+gp}{\PYGZgt{}\PYGZgt{}\PYGZgt{} }\PYG{n}{downloader} \PYG{o}{=} \PYG{n}{ImageDownloader}\PYG{p}{(}\PYG{n}{config\PYGZus{}path}\PYG{o}{=}\PYG{l+s+s1}{\PYGZsq{}}\PYG{l+s+s1}{./custom\PYGZus{}config/config.json}\PYG{l+s+s1}{\PYGZsq{}}\PYG{p}{)}
\end{sphinxVerbatim}
\index{\_\_init\_\_() (in module gebot.ImageDownloader)@\spxentry{\_\_init\_\_()}\spxextra{in module gebot.ImageDownloader}}

\begin{fulllineitems}
\phantomsection\label{\detokenize{index:gebot.ImageDownloader.__init__}}
\pysigstartsignatures
\pysiglinewithargsret{\sphinxcode{\sphinxupquote{gebot.ImageDownloader.}}\sphinxbfcode{\sphinxupquote{\_\_init\_\_}}}{}{}
\pysigstopsignatures
\sphinxAtStartPar
Initialize ImageDownloader class.

\sphinxAtStartPar
The configuration file contains all the required parameters to run this class.
\begin{quote}\begin{description}
\sphinxlineitem{Parameters}
\sphinxAtStartPar
\sphinxstyleliteralstrong{\sphinxupquote{config\_path}} (\sphinxstyleliteralemphasis{\sphinxupquote{str}}\sphinxstyleliteralemphasis{\sphinxupquote{, }}\sphinxstyleliteralemphasis{\sphinxupquote{optional}}) \textendash{} Path to the configuration file (default: ‘./resources/config.json’).

\end{description}\end{quote}

\end{fulllineitems}

\index{download\_image() (in module gebot.ImageDownloader)@\spxentry{download\_image()}\spxextra{in module gebot.ImageDownloader}}

\begin{fulllineitems}
\phantomsection\label{\detokenize{index:gebot.ImageDownloader.download_image}}
\pysigstartsignatures
\pysiglinewithargsret{\sphinxcode{\sphinxupquote{gebot.ImageDownloader.}}\sphinxbfcode{\sphinxupquote{download\_image}}}{\sphinxparam{\DUrole{n,n}{coord}}\sphinxparamcomma \sphinxparam{\DUrole{n,n}{filename}}\sphinxparamcomma \sphinxparam{\DUrole{n,n}{hover\_time}\DUrole{o,o}{=}\DUrole{default_value}{4}}\sphinxparamcomma \sphinxparam{\DUrole{n,n}{step\_sleep}\DUrole{o,o}{=}\DUrole{default_value}{2}}}{}
\pysigstopsignatures
\sphinxAtStartPar
Download an image from input coordinates and save it with the given filename. This method gives more control to the user; the user can download a single image and save it with a user\sphinxhyphen{}defined filename.
\begin{quote}\begin{description}
\sphinxlineitem{Parameters}\begin{itemize}
\item {} 
\sphinxAtStartPar
\sphinxstyleliteralstrong{\sphinxupquote{coord}} (\sphinxstyleliteralemphasis{\sphinxupquote{str}}) \textendash{} Centre coordinates of the image.

\item {} 
\sphinxAtStartPar
\sphinxstyleliteralstrong{\sphinxupquote{filename}} (\sphinxstyleliteralemphasis{\sphinxupquote{str}}) \textendash{} Name of the file to be saved.

\item {} 
\sphinxAtStartPar
\sphinxstyleliteralstrong{\sphinxupquote{hover\_time}} (\sphinxstyleliteralemphasis{\sphinxupquote{int}}\sphinxstyleliteralemphasis{\sphinxupquote{, }}\sphinxstyleliteralemphasis{\sphinxupquote{optional}}) \textendash{} Time to sleep when GE pro is hovering (default: 4).

\item {} 
\sphinxAtStartPar
\sphinxstyleliteralstrong{\sphinxupquote{step\_sleep}} (\sphinxstyleliteralemphasis{\sphinxupquote{int}}\sphinxstyleliteralemphasis{\sphinxupquote{, }}\sphinxstyleliteralemphasis{\sphinxupquote{optional}}) \textendash{} Time to sleep (default: 2).

\end{itemize}

\end{description}\end{quote}

\end{fulllineitems}

\index{download\_images() (in module gebot.ImageDownloader)@\spxentry{download\_images()}\spxextra{in module gebot.ImageDownloader}}

\begin{fulllineitems}
\phantomsection\label{\detokenize{index:gebot.ImageDownloader.download_images}}
\pysigstartsignatures
\pysiglinewithargsret{\sphinxcode{\sphinxupquote{gebot.ImageDownloader.}}\sphinxbfcode{\sphinxupquote{download\_images}}}{\sphinxparam{\DUrole{n,n}{latitude}}\sphinxparamcomma \sphinxparam{\DUrole{n,n}{longitude}}\sphinxparamcomma \sphinxparam{\DUrole{n,n}{img\_id}}\sphinxparamcomma \sphinxparam{\DUrole{n,n}{sleep\_time}\DUrole{o,o}{=}\DUrole{default_value}{0}}\sphinxparamcomma \sphinxparam{\DUrole{n,n}{sleep\_after}\DUrole{o,o}{=}\DUrole{default_value}{25}}}{}
\pysigstopsignatures
\sphinxAtStartPar
Download multiple images from coordinates. This method takes a list of parameters and downloads them; the user has less control over the filename but it is faster.
\begin{quote}\begin{description}
\sphinxlineitem{Parameters}\begin{itemize}
\item {} 
\sphinxAtStartPar
\sphinxstyleliteralstrong{\sphinxupquote{latitude}} (\sphinxstyleliteralemphasis{\sphinxupquote{list}}) \textendash{} List of latitudes.

\item {} 
\sphinxAtStartPar
\sphinxstyleliteralstrong{\sphinxupquote{longitude}} (\sphinxstyleliteralemphasis{\sphinxupquote{list}}) \textendash{} List of longitudes.

\item {} 
\sphinxAtStartPar
\sphinxstyleliteralstrong{\sphinxupquote{img\_id}} (\sphinxstyleliteralemphasis{\sphinxupquote{list}}) \textendash{} List of image IDs.

\item {} 
\sphinxAtStartPar
\sphinxstyleliteralstrong{\sphinxupquote{sleep\_time}} (\sphinxstyleliteralemphasis{\sphinxupquote{int}}\sphinxstyleliteralemphasis{\sphinxupquote{, }}\sphinxstyleliteralemphasis{\sphinxupquote{optional}}) \textendash{} Time to sleep (default: 0).

\item {} 
\sphinxAtStartPar
\sphinxstyleliteralstrong{\sphinxupquote{sleep\_after}} (\sphinxstyleliteralemphasis{\sphinxupquote{int}}\sphinxstyleliteralemphasis{\sphinxupquote{, }}\sphinxstyleliteralemphasis{\sphinxupquote{optional}}) \textendash{} Sleep after a certain number of downloads to avoid banning. (default: 25).

\end{itemize}

\end{description}\end{quote}

\end{fulllineitems}

\index{\_\_check\_download\_complete\_\_() (in module gebot.ImageDownloader)@\spxentry{\_\_check\_download\_complete\_\_()}\spxextra{in module gebot.ImageDownloader}}

\begin{fulllineitems}
\phantomsection\label{\detokenize{index:gebot.ImageDownloader.__check_download_complete__}}
\pysigstartsignatures
\pysiglinewithargsret{\sphinxcode{\sphinxupquote{gebot.ImageDownloader.}}\sphinxbfcode{\sphinxupquote{\_\_check\_download\_complete\_\_}}}{\sphinxparam{\DUrole{n,n}{filename}}}{}
\pysigstopsignatures
\sphinxAtStartPar
Check if the download is complete for a specific file on the given save path. This is an internal method to check the status of the download.
Once the download is completed, it sends a trigger flag to the bot to continue downloading. Additionally, if the bot gets into a ‘stalled’ state and stops downloading, it sends an email to the registered user.
\begin{quote}\begin{description}
\sphinxlineitem{Parameters}
\sphinxAtStartPar
\sphinxstyleliteralstrong{\sphinxupquote{filename}} (\sphinxstyleliteralemphasis{\sphinxupquote{str}}) \textendash{} Name of the file to check.

\end{description}\end{quote}

\end{fulllineitems}

\index{\_\_update\_status\_\_() (in module gebot.ImageDownloader)@\spxentry{\_\_update\_status\_\_()}\spxextra{in module gebot.ImageDownloader}}

\begin{fulllineitems}
\phantomsection\label{\detokenize{index:gebot.ImageDownloader.__update_status__}}
\pysigstartsignatures
\pysiglinewithargsret{\sphinxcode{\sphinxupquote{gebot.ImageDownloader.}}\sphinxbfcode{\sphinxupquote{\_\_update\_status\_\_}}}{}{}
\pysigstopsignatures
\sphinxAtStartPar
Internal method. This method is used to update the status on the notification window of the bot after each image download.

\end{fulllineitems}

\index{\_\_get\_status\_\_() (in module gebot.ImageDownloader)@\spxentry{\_\_get\_status\_\_()}\spxextra{in module gebot.ImageDownloader}}

\begin{fulllineitems}
\phantomsection\label{\detokenize{index:gebot.ImageDownloader.__get_status__}}
\pysigstartsignatures
\pysiglinewithargsret{\sphinxcode{\sphinxupquote{gebot.ImageDownloader.}}\sphinxbfcode{\sphinxupquote{\_\_get\_status\_\_}}}{}{}
\pysigstopsignatures
\sphinxAtStartPar
Internal method for updating the status. This method calculates various variables to be displayed on the status window.
\begin{quote}\begin{description}
\sphinxlineitem{Returns}
\sphinxAtStartPar

\sphinxAtStartPar
A dictionary containing status\sphinxhyphen{}related information.
\begin{itemize}
\item {} 
\sphinxAtStartPar
’status’ (str): The current status of the operation.

\item {} 
\sphinxAtStartPar
’speed’ (str): The processing speed, represented as seconds per image.

\item {} 
\sphinxAtStartPar
’expected\_finish\_time’ (str): The estimated time for completion in days, hours, and minutes.

\item {} 
\sphinxAtStartPar
’remaining\_images’ (str): The number of remaining images to process.

\item {} 
\sphinxAtStartPar
’time\_elapsed’ (str): The time elapsed in days, hours, and minutes since the start of the operation.

\end{itemize}


\sphinxlineitem{Return type}
\sphinxAtStartPar
dict

\end{description}\end{quote}

\begin{sphinxadmonition}{note}{Note:}
\sphinxAtStartPar
The ‘speed’ is calculated as the time taken per image, and ‘expected\_finish\_time’ and ‘time\_elapsed’ are formatted in days, hours, and minutes.
\end{sphinxadmonition}

\end{fulllineitems}

\index{\_\_sec2dhm\_\_() (in module gebot.ImageDownloader)@\spxentry{\_\_sec2dhm\_\_()}\spxextra{in module gebot.ImageDownloader}}

\begin{fulllineitems}
\phantomsection\label{\detokenize{index:gebot.ImageDownloader.__sec2dhm__}}
\pysigstartsignatures
\pysiglinewithargsret{\sphinxcode{\sphinxupquote{gebot.ImageDownloader.}}\sphinxbfcode{\sphinxupquote{\_\_sec2dhm\_\_}}}{\sphinxparam{\DUrole{n,n}{duration\_seconds}}}{}
\pysigstopsignatures
\sphinxAtStartPar
This is an internal helper method to calculate days, hours, and minutes from the input seconds.
\begin{quote}\begin{description}
\sphinxlineitem{Parameters}
\sphinxAtStartPar
\sphinxstyleliteralstrong{\sphinxupquote{duration\_seconds}} (\sphinxstyleliteralemphasis{\sphinxupquote{int}}) \textendash{} Time in seconds.

\sphinxlineitem{Returns}
\sphinxAtStartPar
A list with days, hours, and minutes for input seconds.
\sphinxhyphen{} (days, hours, minutes) (list): Time in Days, Hours, and Minutes

\sphinxlineitem{Return type}
\sphinxAtStartPar
list

\end{description}\end{quote}

\end{fulllineitems}

\phantomsection\label{\detokenize{index:module-getLoc.LocationGetter}}\index{module@\spxentry{module}!getLoc.LocationGetter@\spxentry{getLoc.LocationGetter}}\index{getLoc.LocationGetter@\spxentry{getLoc.LocationGetter}!module@\spxentry{module}}
\sphinxAtStartPar
\sphinxstylestrong{Location Getter}

\sphinxAtStartPar
This class is designed to facilitate the manual retrieval and storage of Google Earth Pro GUI button locations, and update the configuration data in ‘config.json’.
The script serves as a setup script that must be run before configuring GEBOT. It guides the user through the necessary steps to obtain the required configuration file.
\begin{quote}\begin{description}
\sphinxlineitem{param None}
\end{description}\end{quote}
\index{location\_report (in module getLoc.LocationGetter)@\spxentry{location\_report}\spxextra{in module getLoc.LocationGetter}}

\begin{fulllineitems}
\phantomsection\label{\detokenize{index:getLoc.LocationGetter.location_report}}
\pysigstartsignatures
\pysigline{\sphinxcode{\sphinxupquote{getLoc.LocationGetter.}}\sphinxbfcode{\sphinxupquote{location\_report}}}
\pysigstopsignatures
\sphinxAtStartPar
A dictionary to store mouse locations for different elements.
\begin{quote}\begin{description}
\sphinxlineitem{Type}
\sphinxAtStartPar
dict

\end{description}\end{quote}

\end{fulllineitems}

\index{\_\_init\_\_() (in module getLoc.LocationGetter)@\spxentry{\_\_init\_\_()}\spxextra{in module getLoc.LocationGetter}}

\begin{fulllineitems}
\phantomsection\label{\detokenize{index:getLoc.LocationGetter.__init__}}
\pysigstartsignatures
\pysiglinewithargsret{\sphinxcode{\sphinxupquote{getLoc.LocationGetter.}}\sphinxbfcode{\sphinxupquote{\_\_init\_\_}}}{}{}
\pysigstopsignatures
\sphinxAtStartPar
Initializes the LocationGetter.
\begin{quote}\begin{description}
\sphinxlineitem{Parameters}
\sphinxAtStartPar
\sphinxstyleliteralstrong{\sphinxupquote{None}} \textendash{} 

\sphinxlineitem{Return type}
\sphinxAtStartPar
None

\end{description}\end{quote}

\end{fulllineitems}

\index{get\_location() (in module getLoc.LocationGetter)@\spxentry{get\_location()}\spxextra{in module getLoc.LocationGetter}}

\begin{fulllineitems}
\phantomsection\label{\detokenize{index:getLoc.LocationGetter.get_location}}
\pysigstartsignatures
\pysiglinewithargsret{\sphinxcode{\sphinxupquote{getLoc.LocationGetter.}}\sphinxbfcode{\sphinxupquote{get\_location}}}{\sphinxparam{\DUrole{n,n}{message}}}{}
\pysigstopsignatures
\sphinxAtStartPar
Retrieves the mouse location after user input.
\begin{quote}\begin{description}
\sphinxlineitem{Parameters}
\sphinxAtStartPar
\sphinxstyleliteralstrong{\sphinxupquote{message}} (\sphinxstyleliteralemphasis{\sphinxupquote{str}}) \textendash{} The message to display to the user.

\sphinxlineitem{Returns}
\sphinxAtStartPar
x and y coordinates of the mouse position.

\sphinxlineitem{Return type}
\sphinxAtStartPar
tuple

\end{description}\end{quote}

\end{fulllineitems}

\index{collect\_locations() (in module getLoc.LocationGetter)@\spxentry{collect\_locations()}\spxextra{in module getLoc.LocationGetter}}

\begin{fulllineitems}
\phantomsection\label{\detokenize{index:getLoc.LocationGetter.collect_locations}}
\pysigstartsignatures
\pysiglinewithargsret{\sphinxcode{\sphinxupquote{getLoc.LocationGetter.}}\sphinxbfcode{\sphinxupquote{collect\_locations}}}{}{}
\pysigstopsignatures
\sphinxAtStartPar
Collects mouse locations for the search bar, uncheck, save image, and save button.
\begin{quote}\begin{description}
\sphinxlineitem{Parameters}
\sphinxAtStartPar
\sphinxstyleliteralstrong{\sphinxupquote{None}} \textendash{} 

\end{description}\end{quote}

\end{fulllineitems}

\index{print\_location\_report() (in module getLoc.LocationGetter)@\spxentry{print\_location\_report()}\spxextra{in module getLoc.LocationGetter}}

\begin{fulllineitems}
\phantomsection\label{\detokenize{index:getLoc.LocationGetter.print_location_report}}
\pysigstartsignatures
\pysiglinewithargsret{\sphinxcode{\sphinxupquote{getLoc.LocationGetter.}}\sphinxbfcode{\sphinxupquote{print\_location\_report}}}{}{}
\pysigstopsignatures
\sphinxAtStartPar
Saves the location report as a JSON file and prints the location report.
\begin{quote}\begin{description}
\sphinxlineitem{Parameters}
\sphinxAtStartPar
\sphinxstyleliteralstrong{\sphinxupquote{None}} \textendash{} 

\end{description}\end{quote}

\end{fulllineitems}

\phantomsection\label{\detokenize{index:module-notificationHandler.SendEmail}}\index{module@\spxentry{module}!notificationHandler.SendEmail@\spxentry{notificationHandler.SendEmail}}\index{notificationHandler.SendEmail@\spxentry{notificationHandler.SendEmail}!module@\spxentry{module}}
\sphinxAtStartPar
\sphinxstylestrong{SendEmail}

\sphinxAtStartPar
A class to send email notifications when the GE bot process is stopped.
\index{\_\_init\_\_() (in module notificationHandler.SendEmail)@\spxentry{\_\_init\_\_()}\spxextra{in module notificationHandler.SendEmail}}

\begin{fulllineitems}
\phantomsection\label{\detokenize{index:notificationHandler.SendEmail.__init__}}
\pysigstartsignatures
\pysiglinewithargsret{\sphinxcode{\sphinxupquote{notificationHandler.SendEmail.}}\sphinxbfcode{\sphinxupquote{\_\_init\_\_}}}{}{}
\pysigstopsignatures
\sphinxAtStartPar
Initializes the SendEmail object.
\begin{quote}\begin{description}
\sphinxlineitem{Parameters}\begin{itemize}
\item {} 
\sphinxAtStartPar
\sphinxstyleliteralstrong{\sphinxupquote{config\_path}} (\sphinxstyleliteralemphasis{\sphinxupquote{str}}\sphinxstyleliteralemphasis{\sphinxupquote{, }}\sphinxstyleliteralemphasis{\sphinxupquote{optional}}) \textendash{} The path to the configuration file (default is ‘./resources/config.json’).

\item {} 
\sphinxAtStartPar
\sphinxstyleliteralstrong{\sphinxupquote{credData}} (\sphinxstyleliteralemphasis{\sphinxupquote{str}}\sphinxstyleliteralemphasis{\sphinxupquote{, }}\sphinxstyleliteralemphasis{\sphinxupquote{optional}}) \textendash{} The path to the credential data file (default is “./resources/cred.dat”).

\end{itemize}

\sphinxlineitem{Return type}
\sphinxAtStartPar
None

\end{description}\end{quote}

\end{fulllineitems}

\index{process\_stopped() (in module notificationHandler.SendEmail)@\spxentry{process\_stopped()}\spxextra{in module notificationHandler.SendEmail}}

\begin{fulllineitems}
\phantomsection\label{\detokenize{index:notificationHandler.SendEmail.process_stopped}}
\pysigstartsignatures
\pysiglinewithargsret{\sphinxcode{\sphinxupquote{notificationHandler.SendEmail.}}\sphinxbfcode{\sphinxupquote{process\_stopped}}}{}{}
\pysigstopsignatures
\sphinxAtStartPar
Sends email notifications when the GE bot process is stopped.

\sphinxAtStartPar
For each email ID in the configured list, connects to the SMTP server, logs in, and sends a notification email.
\begin{quote}\begin{description}
\sphinxlineitem{Parameters}
\sphinxAtStartPar
\sphinxstyleliteralstrong{\sphinxupquote{None}} \textendash{} 

\sphinxlineitem{Return type}
\sphinxAtStartPar
None

\end{description}\end{quote}

\end{fulllineitems}

\phantomsection\label{\detokenize{index:module-statusIndicator.GEBotInfoDisplay}}\index{module@\spxentry{module}!statusIndicator.GEBotInfoDisplay@\spxentry{statusIndicator.GEBotInfoDisplay}}\index{statusIndicator.GEBotInfoDisplay@\spxentry{statusIndicator.GEBotInfoDisplay}!module@\spxentry{module}}
\sphinxAtStartPar
\sphinxstylestrong{GEBotInfoDisplay}

\sphinxAtStartPar
A class for creating and updating a GUI display for GE Bot information using tkinter.
\index{\_\_init\_\_() (in module statusIndicator.GEBotInfoDisplay)@\spxentry{\_\_init\_\_()}\spxextra{in module statusIndicator.GEBotInfoDisplay}}

\begin{fulllineitems}
\phantomsection\label{\detokenize{index:statusIndicator.GEBotInfoDisplay.__init__}}
\pysigstartsignatures
\pysiglinewithargsret{\sphinxcode{\sphinxupquote{statusIndicator.GEBotInfoDisplay.}}\sphinxbfcode{\sphinxupquote{\_\_init\_\_}}}{}{}
\pysigstopsignatures
\sphinxAtStartPar
Initializes the GEBotInfoDisplay object with the specified Tkinter root window.
\begin{quote}\begin{description}
\sphinxlineitem{Parameters}
\sphinxAtStartPar
\sphinxstyleliteralstrong{\sphinxupquote{(}}\sphinxstyleliteralstrong{\sphinxupquote{tk.Tk}}\sphinxstyleliteralstrong{\sphinxupquote{)}} (\sphinxstyleliteralemphasis{\sphinxupquote{root}}) \textendash{} The Tkinter root window to which the display will be attached.

\end{description}\end{quote}

\end{fulllineitems}

\index{create\_label() (in module statusIndicator.GEBotInfoDisplay)@\spxentry{create\_label()}\spxextra{in module statusIndicator.GEBotInfoDisplay}}

\begin{fulllineitems}
\phantomsection\label{\detokenize{index:statusIndicator.GEBotInfoDisplay.create_label}}
\pysigstartsignatures
\pysiglinewithargsret{\sphinxcode{\sphinxupquote{statusIndicator.GEBotInfoDisplay.}}\sphinxbfcode{\sphinxupquote{create\_label}}}{}{}
\pysigstopsignatures
\sphinxAtStartPar
Creates a labeled display area for a specific information category.
\begin{quote}\begin{description}
\sphinxlineitem{Parameters}\begin{itemize}
\item {} 
\sphinxAtStartPar
\sphinxstyleliteralstrong{\sphinxupquote{text}} (\sphinxstyleliteralemphasis{\sphinxupquote{str}}) \textendash{} The label text.

\item {} 
\sphinxAtStartPar
\sphinxstyleliteralstrong{\sphinxupquote{color}} (\sphinxstyleliteralemphasis{\sphinxupquote{str}}) \textendash{} The color of the label text.

\item {} 
\sphinxAtStartPar
\sphinxstyleliteralstrong{\sphinxupquote{variable}} (\sphinxstyleliteralemphasis{\sphinxupquote{tk.StringVar}}) \textendash{} The Tkinter StringVar associated with the displayed information.

\end{itemize}

\end{description}\end{quote}

\end{fulllineitems}

\index{update\_info() (in module statusIndicator.GEBotInfoDisplay)@\spxentry{update\_info()}\spxextra{in module statusIndicator.GEBotInfoDisplay}}

\begin{fulllineitems}
\phantomsection\label{\detokenize{index:statusIndicator.GEBotInfoDisplay.update_info}}
\pysigstartsignatures
\pysiglinewithargsret{\sphinxcode{\sphinxupquote{statusIndicator.GEBotInfoDisplay.}}\sphinxbfcode{\sphinxupquote{update\_info}}}{}{}
\pysigstopsignatures
\sphinxAtStartPar
Updates the displayed information based on the provided status dictionary.
:param status\_dic: A dictionary containing information about GE Bot status.
:type status\_dic: dict

\end{fulllineitems}

\phantomsection\label{\detokenize{index:module-georef.Geotagger}}\index{module@\spxentry{module}!georef.Geotagger@\spxentry{georef.Geotagger}}\index{georef.Geotagger@\spxentry{georef.Geotagger}!module@\spxentry{module}}
\sphinxAtStartPar
\sphinxstylestrong{Geotagger}

\sphinxAtStartPar
A class for geotagging images and saving them in TIFF format with spatial information. This should be used only to the images downloaded via GEBOT.

\begin{sphinxadmonition}{note}{Note:}
\sphinxAtStartPar
Geotagger class should be run using main function. It can take PNG/JPG files and georeference them to TIFF format.
\end{sphinxadmonition}
\begin{quote}\begin{description}
\sphinxlineitem{param inputpath}
\sphinxAtStartPar
Path to the image folder.

\sphinxlineitem{type inputpath}
\sphinxAtStartPar
str/path

\sphinxlineitem{param outputpath}
\sphinxAtStartPar
Path of the folder where the georeferenced images should be saved. If set to ‘None’, a new folder will be created in the imagePath with ‘\_GEOTAGGED’ appended to the path.

\sphinxlineitem{type outputpath}
\sphinxAtStartPar
str/path

\sphinxlineitem{param start}
\sphinxAtStartPar
The starting number for the filenames of the images in the folder for georeferencing. If set to ‘None’, the function will start from the beginning (start=0).

\sphinxlineitem{type start}
\sphinxAtStartPar
int

\sphinxlineitem{param stop}
\sphinxAtStartPar
The ending number for the filenames of the images in the folder for georeferencing. If set to ‘None’, the function will perform the action for all the items in the folder.

\sphinxlineitem{type stop}
\sphinxAtStartPar
int

\end{description}\end{quote}

\sphinxAtStartPar
The ‘start’ and ‘stop’ parameters are helpful if we want to terminate the process in the middle and restart it later.
\subsubsection*{Examples}

\begin{sphinxVerbatim}[commandchars=\\\{\}]
\PYG{g+gp}{\PYGZgt{}\PYGZgt{}\PYGZgt{} }\PYG{n}{python} \PYG{n}{georef}\PYG{o}{.}\PYG{n}{py} \PYG{o}{\PYGZhy{}}\PYG{o}{\PYGZhy{}}\PYG{n}{inputpath} \PYG{n}{path}\PYG{o}{/}\PYG{n}{to}\PYG{o}{/}\PYG{n}{image\PYGZus{}folder} \PYG{o}{\PYGZhy{}}\PYG{o}{\PYGZhy{}}\PYG{n}{outputpath} \PYG{n}{path}\PYG{o}{/}\PYG{n}{to}\PYG{o}{/}\PYG{n}{save\PYGZus{}folder} \PYG{o}{\PYGZhy{}}\PYG{o}{\PYGZhy{}}\PYG{n}{start} \PYG{l+m+mi}{10} \PYG{o}{\PYGZhy{}}\PYG{o}{\PYGZhy{}}\PYG{n}{stop} \PYG{l+m+mi}{20}
\end{sphinxVerbatim}
\index{\_\_init\_\_() (in module georef.Geotagger)@\spxentry{\_\_init\_\_()}\spxextra{in module georef.Geotagger}}

\begin{fulllineitems}
\phantomsection\label{\detokenize{index:georef.Geotagger.__init__}}
\pysigstartsignatures
\pysiglinewithargsret{\sphinxcode{\sphinxupquote{georef.Geotagger.}}\sphinxbfcode{\sphinxupquote{\_\_init\_\_}}}{}{}
\pysigstopsignatures
\sphinxAtStartPar
Initializes the Geotagger object.
\begin{quote}\begin{description}
\sphinxlineitem{Parameters}\begin{itemize}
\item {} 
\sphinxAtStartPar
\sphinxstyleliteralstrong{\sphinxupquote{filepath}} (\sphinxstyleliteralemphasis{\sphinxupquote{str}}) \textendash{} The path to the image file.

\item {} 
\sphinxAtStartPar
\sphinxstyleliteralstrong{\sphinxupquote{center\_coord}} (\sphinxstyleliteralemphasis{\sphinxupquote{tuple}}) \textendash{} Tuple containing latitude and longitude values of the image center.

\item {} 
\sphinxAtStartPar
\sphinxstyleliteralstrong{\sphinxupquote{savepath}} (\sphinxstyleliteralemphasis{\sphinxupquote{str}}) \textendash{} The path to the folder where geotagged images will be saved.

\item {} 
\sphinxAtStartPar
\sphinxstyleliteralstrong{\sphinxupquote{pixXRES}} (\sphinxstyleliteralemphasis{\sphinxupquote{float}}) \textendash{} Pixel resolution in the X direction.

\item {} 
\sphinxAtStartPar
\sphinxstyleliteralstrong{\sphinxupquote{pixYRES}} (\sphinxstyleliteralemphasis{\sphinxupquote{float}}) \textendash{} Pixel resolution in the Y direction.

\end{itemize}

\sphinxlineitem{Return type}
\sphinxAtStartPar
None

\end{description}\end{quote}

\end{fulllineitems}

\index{lat\_long() (in module georef.Geotagger)@\spxentry{lat\_long()}\spxextra{in module georef.Geotagger}}

\begin{fulllineitems}
\phantomsection\label{\detokenize{index:georef.Geotagger.lat_long}}
\pysigstartsignatures
\pysiglinewithargsret{\sphinxcode{\sphinxupquote{georef.Geotagger.}}\sphinxbfcode{\sphinxupquote{lat\_long}}}{}{}
\pysigstopsignatures
\sphinxAtStartPar
This is a local method to calculates new latitude and longitude with the given offsets.
\begin{quote}\begin{description}
\sphinxlineitem{Parameters}\begin{itemize}
\item {} 
\sphinxAtStartPar
\sphinxstyleliteralstrong{\sphinxupquote{lat}} (\sphinxstyleliteralemphasis{\sphinxupquote{float}}) \textendash{} Latitude.

\item {} 
\sphinxAtStartPar
\sphinxstyleliteralstrong{\sphinxupquote{lon}} (\sphinxstyleliteralemphasis{\sphinxupquote{float}}) \textendash{} Longitude.

\item {} 
\sphinxAtStartPar
\sphinxstyleliteralstrong{\sphinxupquote{dn}} (\sphinxstyleliteralemphasis{\sphinxupquote{float}}) \textendash{} Offset in the north direction.

\item {} 
\sphinxAtStartPar
\sphinxstyleliteralstrong{\sphinxupquote{de}} (\sphinxstyleliteralemphasis{\sphinxupquote{float}}) \textendash{} Offset in the east direction.

\end{itemize}

\sphinxlineitem{Returns}
\sphinxAtStartPar
Tuple containing new latitude, longitude, and altitude (0).

\sphinxlineitem{Return type}
\sphinxAtStartPar
tuple

\end{description}\end{quote}

\end{fulllineitems}

\index{output\_corners() (in module georef.Geotagger)@\spxentry{output\_corners()}\spxextra{in module georef.Geotagger}}

\begin{fulllineitems}
\phantomsection\label{\detokenize{index:georef.Geotagger.output_corners}}
\pysigstartsignatures
\pysiglinewithargsret{\sphinxcode{\sphinxupquote{georef.Geotagger.}}\sphinxbfcode{\sphinxupquote{output\_corners}}}{}{}
\pysigstopsignatures
\sphinxAtStartPar
Computes the corners of the geotagged image based on center coordinates and pixel resolutions.
\begin{quote}\begin{description}
\sphinxlineitem{Parameters}\begin{itemize}
\item {} 
\sphinxAtStartPar
\sphinxstyleliteralstrong{\sphinxupquote{lat}} (\sphinxstyleliteralemphasis{\sphinxupquote{float}}) \textendash{} Latitude of the image center.

\item {} 
\sphinxAtStartPar
\sphinxstyleliteralstrong{\sphinxupquote{lon}} (\sphinxstyleliteralemphasis{\sphinxupquote{float}}) \textendash{} Longitude of the image center.

\item {} 
\sphinxAtStartPar
\sphinxstyleliteralstrong{\sphinxupquote{width}} (\sphinxstyleliteralemphasis{\sphinxupquote{int}}) \textendash{} Width of the image in pixels.

\item {} 
\sphinxAtStartPar
\sphinxstyleliteralstrong{\sphinxupquote{height}} (\sphinxstyleliteralemphasis{\sphinxupquote{int}}) \textendash{} Height of the image in pixels.

\item {} 
\sphinxAtStartPar
\sphinxstyleliteralstrong{\sphinxupquote{offset1}} (\sphinxstyleliteralemphasis{\sphinxupquote{float}}) \textendash{} Offset in the north direction.

\item {} 
\sphinxAtStartPar
\sphinxstyleliteralstrong{\sphinxupquote{offset2}} (\sphinxstyleliteralemphasis{\sphinxupquote{float}}) \textendash{} Offset in the east direction.

\end{itemize}

\sphinxlineitem{Returns}
\sphinxAtStartPar
Array containing the coordinates of the image corners.

\sphinxlineitem{Return type}
\sphinxAtStartPar
numpy.ndarray

\end{description}\end{quote}

\end{fulllineitems}

\index{name2latlong() (in module georef.Geotagger)@\spxentry{name2latlong()}\spxextra{in module georef.Geotagger}}

\begin{fulllineitems}
\phantomsection\label{\detokenize{index:georef.Geotagger.name2latlong}}
\pysigstartsignatures
\pysiglinewithargsret{\sphinxcode{\sphinxupquote{georef.Geotagger.}}\sphinxbfcode{\sphinxupquote{name2latlong}}}{}{}
\pysigstopsignatures
\sphinxAtStartPar
Extracts latitude and longitude from the image filename.
\begin{quote}\begin{description}
\sphinxlineitem{Parameters}
\sphinxAtStartPar
\sphinxstyleliteralstrong{\sphinxupquote{filename}} (\sphinxstyleliteralemphasis{\sphinxupquote{str}}) \textendash{} The name of the image file.

\sphinxlineitem{Returns}
\sphinxAtStartPar
Tuple containing latitude and longitude.

\sphinxlineitem{Return type}
\sphinxAtStartPar
tuple

\end{description}\end{quote}

\end{fulllineitems}

\index{getcoord() (in module georef.Geotagger)@\spxentry{getcoord()}\spxextra{in module georef.Geotagger}}

\begin{fulllineitems}
\phantomsection\label{\detokenize{index:georef.Geotagger.getcoord}}
\pysigstartsignatures
\pysiglinewithargsret{\sphinxcode{\sphinxupquote{georef.Geotagger.}}\sphinxbfcode{\sphinxupquote{getcoord}}}{}{}
\pysigstopsignatures
\sphinxAtStartPar
Computes the geotagged image’s bounding box and corner coordinates.
\begin{quote}\begin{description}
\sphinxlineitem{Parameters}
\sphinxAtStartPar
\sphinxstyleliteralstrong{\sphinxupquote{None}} \textendash{} 

\sphinxlineitem{Returns}
\sphinxAtStartPar
Tuple containing the bounding box coordinates (north, south, west, east) and the image.

\sphinxlineitem{Return type}
\sphinxAtStartPar
tuple

\end{description}\end{quote}

\end{fulllineitems}

\index{geotag() (in module georef.Geotagger)@\spxentry{geotag()}\spxextra{in module georef.Geotagger}}

\begin{fulllineitems}
\phantomsection\label{\detokenize{index:georef.Geotagger.geotag}}
\pysigstartsignatures
\pysiglinewithargsret{\sphinxcode{\sphinxupquote{georef.Geotagger.}}\sphinxbfcode{\sphinxupquote{geotag}}}{}{}
\pysigstopsignatures
\sphinxAtStartPar
Geotags the image and saves it in TIFF format.

\end{fulllineitems}



\renewcommand{\indexname}{Python Module Index}
\begin{sphinxtheindex}
\let\bigletter\sphinxstyleindexlettergroup
\bigletter{g}
\item\relax\sphinxstyleindexentry{gebot.ImageDownloader}\sphinxstyleindexpageref{index:\detokenize{module-gebot.ImageDownloader}}
\item\relax\sphinxstyleindexentry{georef.Geotagger}\sphinxstyleindexpageref{index:\detokenize{module-georef.Geotagger}}
\item\relax\sphinxstyleindexentry{getLoc.LocationGetter}\sphinxstyleindexpageref{index:\detokenize{module-getLoc.LocationGetter}}
\indexspace
\bigletter{n}
\item\relax\sphinxstyleindexentry{notificationHandler.SendEmail}\sphinxstyleindexpageref{index:\detokenize{module-notificationHandler.SendEmail}}
\indexspace
\bigletter{s}
\item\relax\sphinxstyleindexentry{statusIndicator.GEBotInfoDisplay}\sphinxstyleindexpageref{index:\detokenize{module-statusIndicator.GEBotInfoDisplay}}
\end{sphinxtheindex}

\renewcommand{\indexname}{Index}
\printindex
\end{document}